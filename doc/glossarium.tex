\newpage
\section{Глоссарий}

\begin{itemize}
  \item визуальное представление -- интерпретация числовой и текстовой информации в 
    виде графиков, структурных схем, таблиц, карт и изображений;
  \item ГП (или GPU) -- графический процессор;
  \item ЦП -- центральный процессор;
  \item шейдер -- программа, исполняемая на графическом процессоре;
  \item фрагмент -- участок размером с пиксель;
  \item проход (в графическом конвейере) -- проход данных по графическому конвейеру;
  \item селектор (кадрового буфера) -- параметр, указывающий какой из двух кадровых буферов
    необходимо вернуть;
  \item вершинный буфер -- особенность OpenGL, обеспечивающая загрузку данных на видеоустройство;
  \item кадровый буфер -- область памяти видеокарты для хранения изображения;
  \item частица -- термин, который используется для обозначения объектов, которые в 
    контексте физической симуляции можно считать неделимыми;
  \item SPH (англ. Smoothed Particle Hydronynamics) -- гидродинамика сглаженных частиц. 
    Вычислительный метод для симуляции жидкостей и газов;
  \item UV-координата -- двумерный вектор, обозначающий положение на текстуре.
\end{itemize}
