\newpage
\section{ГЛОССАРИЙ}

\begin{itemize}
  \item Визуальное представление -- интерпретация числовой и текстовой информации в 
    виде графиков, структурных схем, таблиц, карт и изображений.
  \item ГП (или GPU) -- графический процессор.
  \item ЦП -- центральный процессор.
  \item Шейдер -- программа, исполняемая на графическом процессоре.
  \item Фрагмент -- участок размером с пиксель.
  \item Проход (в графическом конвейере) -- проход данных по графическому конвейеру.
  \item Селектор (кадрового буфера) -- параметр, указывающий какой из двух кадровых буферов
    необходимо вернуть.
  \item Вершинный буфер -- особенность OpenGL, обеспечивающая загрузку данных на видеоустройство.
  \item Кадровый буфер -- область памяти видеокарты для хранения изображения.
  \item Частица -- термин, который используется для обозначения объектов, которые в 
    контексте физической симуляции можно считать неделимыми.
  \item SPH (англ. Smoothed Particle Hydronynamics) -- гидродинамика сглаженных частиц. 
    Вычислительный метод для симуляции жидкостей и газов.
  \item UV-координата -- двумерный вектор, обозначающий положение на текстуре.
\end{itemize}
