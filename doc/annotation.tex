\newpage
\section*{Аннотация}

\hfill
\begin{minipage}[t]{0.62\textwidth}
  \ESKDtheAuthor \space \ESKDtheDocName -- Челябинск ЮУрГУ, ПС, 2013. -- \ESKDtotal{page} с., 
  10~ил., библиогр. список~--~\ESKDtotal{bibitem}, 6 листов чертежей ф. А1
\end{minipage} \\
\\

В данной работе рассмотрена разработка инструмента для создания визуальных
представлений, требующих параллельных вычислений на большом объеме данных, 
в веб-приложениях.

Целью дипломной работы явилось разработка программного инструмента для эффективного
создания визуальных представлений в веб-приложениях. Рассмотренные типы
визуализаций требуют особого подхода к вычислениям, так как отображают
взаимодействие большого количества частиц (от десяти тысяч) в реальном времени.
Подобный подход к вычислениям может быть использован для симуляции природных явлений 
и вычислении операций на большом количестве данных. Для достижения цели были поставлены
задачи: изучить способы визуализации данных в веб-приложениях с целью поиска эффективного
способа отображения, создание инструмента для визуализации и реализация демонстрационного 
примера для тестирования производительности.

Ожидаемые результаты:

\begin{itemize}
  \item эффективный способ визуализации сложных процессов в \\веб-приложениях;
  \item уменьшение времени на разработку подобных визуальных представлений.
\end{itemize}

Область применения -- обучающие сервисы, научные статьи и экспертные системы.
