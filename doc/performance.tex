\newpage
\section{Тестирование}

\subsection{Производительность}

Все тесты проводились на демонстрационном примере с различной конфигурацией
симуляции. Основной характеристикой производительности является количество 
кадров в секунду, генерируемых визуализацией.

Для вычисления количества кадров в секунду используются функции стандартной 
библиотеки javascript. Данные снимаются по минимальным значениям показателей.
Для чистоты эксперимента сравнения проводились на разных компьютерах. \\

Конфигурация первого компьютера:

\begin{itemize}
  \item видеокарта -- Intel HD 4000;
  \item процессор -- Intel Core i5;
  \item объем ОЗУ -- 4 гб.
\end{itemize}

Результаты вычислений с различными типами визуализации приведены 
в таблицах \ref{tab:fst:simple}. По горизонтали изменяется количество частиц.
По вертикали тип визуализации.

Одним из параметров, который существенно влияет на скорость визуализации -- это
радиус взаимодействия частиц. Чем он больше, тем большее количество частиц необходимо
обработать. Размер пространства $150\times150$. Радиус указывается в измерениях
виртуального пространства. Результаты тестирования приведены в таблице \ref{tab:fst:radius}.

Для демонстрации производительности алгоритма поиска ближайших, проведены
тесты с полным перебором. Результаты приведены в таблице \ref{tab:fst:algorithm}.

В целях демонстрации, реализован алгоритм просчета визуализации на центральном процессоре.
При этом время, затраченное на отображение, не учитывается. Результаты тестирования 
приведены в таблице \ref{tab:fst:cpu}.  \\

Конфигурация второго компьютера:

\begin{itemize} 
  \item видеокарта -- Nvidia GeForce 640;
  \item процессор -- Intel Core i7;
  \item объем ОЗУ -- 8 гб.
\end{itemize} 

Для второго компьютера проведены те же тесты. Результаты приведены в таблицах
\ref{tab:snd:simple}, \ref{tab:snd:radius}, \ref{tab:snd:algorithm}, \ref{tab:snd:cpu}.

\subsection{Совместимость}

Для тестирования совместимости были выбраны самые популярные браузеры и
проверена работоспособность в различных версиях. 

Для сравнения были отобраны следующие браузеры:

\begin{itemize}
  \item Internet Explorer -- результаты приведены в таблице \ref{tab:brows:ie};
  \item Firefox -- результаты приведены в таблице \ref{tab:brows:ff};
  \item Chrome -- результаты приведены в таблице \ref{tab:brows:chrome};
  \item Safari -- результаты приведены в таблице \ref{tab:brows:safari};
  \item Opera -- результаты приведены в таблице \ref{tab:brows:opera}.
\end{itemize}

На мобильных браузерах данный инструмент не работает ввиду отсутствия поддержки
WebGL.


% написать как проводились вычисления fps
% различная визуализация
% n^2 и поиск ближайших
% сравнение по количеству частиц
% сравнение на разных gpu
% сравнение cpu и gpu
