\newpage
\section{Тестирование}

\subsection{Производительность}

Все тесты проводились на демонстрационном примере с различной конфигурацией
симуляции. Основной характеристикой производительности является количество 
кадров в секунду, генерируемых визуализацией.

Для вычисления количества кадров в секунду используются функции стандартной 
библиотеки javascript. Данные снимаются по минимальным значениям показателей.
Для чистоты эксперимента сравнения проводились на разных компьютерах. \\

Конфигурация первого компьютера:

\begin{itemize}
  \item видеокарта -- Intel HD 4000;
  \item процессор -- Intel Core i5;
  \item объем ОЗУ -- 4 гб.
\end{itemize}

Результаты вычислений с различными типами визуализации приведены 
в таблицах \ref{tab:fst:simple}. По горизонтали изменяется количество частиц.
По вертикали тип визуализации.

\begin{table}[H]
  \caption{\label{tab:fst:simple}Комп. №1. Зависимость от типа визуализации}
  \begin{center}
    \begin{tabular}{|l|c|c|c|c|c|}
      \hline
      Тип визуализации & 100 ч. & 1000 ч. & 5000 ч. & 10000 ч. & 100000 ч. \\
      \hline
      Облако точек & 320 & 175 & 97 & 40 & 24 \\
      Направленные векторы & 320 & 170 & 95 & 39 & 23 \\
      \hline
    \end{tabular}
  \end{center}
\end{table}

Одним из параметров, который существенно влияет на скорость визуализации -- это
радиус взаимодействия частиц. Чем он больше, тем большее количество частиц необходимо
обработать. Размер пространства $150\times150$. Радиус указывается в измерениях
виртуального пространства. Результаты тестирования приведены в таблице \ref{tab:fst:radius}.

\begin{table}[H]
  \caption{\label{tab:fst:radius}Комп. №1. Зависимость от радиуса взаимодействия}
  \begin{center}
    \begin{tabular}{|c|c|c|c|c|c|}
      \hline
      Радиус & 100 ч. & 1000 ч. & 5000 ч. & 10000 ч. & 100000 ч. \\
      \hline
      5 & 320 & 175 & 97 & 40 & 24 \\
      10 & 250 & 120 & 84 & 37 & 22 \\
      50 & 100 & 90 & 43 & 20 & 8 \\
      \hline
    \end{tabular}
  \end{center}
\end{table}

Для демонстрации производительности алгоритма поиска ближайших, проведены
тесты с полным перебором. Результаты приведены в таблице \ref{tab:fst:algorithm}.

\begin{table}[H]
  \caption{\label{tab:fst:algorithm}Комп. №1. Зависимость от алгоритма поиска ближайших частиц}
  \begin{center}
    \begin{tabular}{|c|c|c|c|c|c|}
      \hline
      Алгоритм & 100 ч. & 1000 ч. & 5000 ч. & 10000 ч. & 100000 ч. \\
      \hline
      Использование сетки & 320 & 175 & 97 & 40 & 24 \\
      Полный перебор & 343 & 170 & 70 & 20 & 2 \\
      \hline
    \end{tabular}
  \end{center}
\end{table}

В целях демонстрации, реализован алгоритм просчета визуализации на центральном процессоре.
При этом время, затраченное на отображение, не учитывается. Результаты тестирования 
приведены в таблице \ref{tab:fst:cpu}.  \\

\begin{table}[H] 
  \caption{\label{tab:fst:cpu}Комп. №1. Зависимость от типа процессора} 
  \begin{center} 
    \begin{tabular}{|l|c|c|c|c|c|} 
      \hline
      Тип процессора & 100 ч. & 1000 ч. & 5000 ч. & 10000 ч. & 100000 ч. \\
      \hline
      Графический & 320 & 175 & 97 & 40 & 24 \\
      Центральный & 140 & 100 & 30 & 7 & $\approx{}0.3$ \\
      \hline
    \end{tabular} 
  \end{center} 
\end{table} 

Конфигурация второго компьютера:

\begin{itemize} 
  \item видеокарта -- Nvidia GeForce 640;
  \item процессор -- Intel Core i7;
  \item объем ОЗУ -- 8 гб.
\end{itemize} 

Для второго компьютера проведены те же тесты. Результаты приведены в таблицах
\ref{tab:snd:simple}, \ref{tab:snd:radius}, \ref{tab:snd:algorithm}, \ref{tab:snd:cpu}.

\begin{table}[H] 
  \caption{\label{tab:snd:simple}Комп. №2. Зависимость от типа визуализации} 
  \begin{center}
    \begin{tabular}{|c|c|c|c|c|c|}
      \hline
      Тип визуализации & 100 ч. & 1000 ч. & 5000 ч. & 10000 ч. & 100000 ч. \\
      \hline
      Облако точек & 398 & 315 & 150 & 64 & 47 \\
      Направленные векторы & 399 & 310 & 152 & 63 & 46 \\
      \hline
    \end{tabular}
  \end{center}
\end{table}

\begin{table}[H]
  \caption{\label{tab:snd:radius}Комп. №2. Зависимость от радиуса взаимодействия}
  \begin{center}
    \begin{tabular}{|c|c|c|c|c|c|}
      \hline
      Радиус & 100 ч. & 1000 ч. & 5000 ч. & 10000 ч. & 100000 ч. \\
      \hline
      5 & 398 & 315 & 150 & 64 & 47 \\
      10 & 330 & 205 & 130 & 37 & 22 \\
      50 & 340 & 110 & 78 & 20 & 11 \\
      \hline
    \end{tabular}
  \end{center}
\end{table}

\begin{table}[H]
  \caption{\label{tab:snd:algorithm}Комп. №3. Зависимость от алгоритма поиска ближайших частиц}
  \begin{center}
    \begin{tabular}{|c|c|c|c|c|c|}
      \hline
      Алгоритм & 100 ч. & 1000 ч. & 5000 ч. & 10000 ч. & 100000 ч. \\
      \hline
      Использование сетки & 398 & 315 & 150 & 64 & 47 \\
      Полный перебор & 324 & 203 & 121 & 70 & 8 \\
      \hline
    \end{tabular}
  \end{center}
\end{table}

\begin{table}[H]
  \caption{\label{tab:snd:cpu}Комп. №2. Зависимость от типы процессора}
  \begin{center}
    \begin{tabular}{|c|c|c|c|c|c|}
      \hline
      Тип процессора & 100 ч. & 1000 ч. & 5000 ч. & 10000 ч. & 100000 ч. \\
      \hline
      Графический & 398 & 315 & 150 & 64 & 47 \\
      Центральный & 153 & 102 & 31 & 7 & $\approx{}0.4$ \\
      \hline
    \end{tabular}
  \end{center}
\end{table}

\subsection{Совместимость}

Для тестирования совместимости были выбраны самые популярные браузеры и
проверена работоспособность в различных версиях. 

Для сравнения были отобраны следующие браузеры:

\begin{itemize}
  \item Internet Explorer -- результаты приведены в таблице \ref{tab:brows:ie};
  \item Firefox -- результаты приведены в таблице \ref{tab:brows:ff};
  \item Chrome -- результаты приведены в таблице \ref{tab:brows:chrome};
  \item Safari -- результаты приведены в таблице \ref{tab:brows:safari};
  \item Opera -- результаты приведены в таблице \ref{tab:brows:opera}.
\end{itemize}

На мобильных браузерах данный инструмент не работает ввиду отсутствия поддержки
WebGL.

\begin{table}[h!]
  \caption{\label{tab:brows:ie}Сравнение версий Internet Explorer}
  \begin{center}
    \begin{tabular}{|c|c|l|}
      \hline
      Версия & Дата релиза & Статус \\
      \hline
      11.0 & Октябрь 2013 & Работает как ожидалось \\
      < 10.0 & Октябрь 2012 & Не поддерживается \\
      \hline
    \end{tabular}
  \end{center}
\end{table}

\begin{table}[h!]
  \caption{\label{tab:brows:ff}Сравнение версий Firefox}
  \begin{center}
    \begin{tabular}{|c|c|l|}
      \hline
      Версия & Дата релиза & Статус \\
      \hline
      30.0 & Июнь 2014 & Работает как ожидалось \\
      15.0 & Август 2012 & Работает как ожидалось \\
      10.0 & Январь 2012 & Незначительное падение FPS \\
      < 3.6 & Январь 2010 & Не поддерживается \\
      \hline
    \end{tabular}
  \end{center}
\end{table}

\begin{table}[h!]
  \caption{\label{tab:brows:chrome}Сравнение версий Chrome}
  \begin{center}
    \begin{tabular}{|c|c|l|}
      \hline
      Версия & Дата релиза & Статус \\
      \hline
      35.0 & Май 2014 & Работает как ожидалось \\
      18.0 & Март 2012 & Работает как ожидалось \\
      10.0 & Март 2011 & Работает как ожидалось \\
      < 7.0 & Октябрь 2010 & Не поддерживается \\
      \hline
    \end{tabular}
  \end{center}
\end{table}

\begin{table}[h!]
  \caption{\label{tab:brows:safari}Сравнение версий Safari}
  \begin{center}
    \begin{tabular}{|c|c|l|}
      \hline
      Версия & Дата релиза & Статус \\
      \hline
      7.0 & Май 2014 & Работает как ожидалось \\
      5.1 & Сентябрь 2013 & Работает как ожидалось \\
      < 5.0 & Июль 2011 & Не поддерживается \\
      \hline
    \end{tabular}
  \end{center}
\end{table}

\begin{table}[h!]
  \caption{\label{tab:brows:opera}Сравнение версий Opera}
  \begin{center}
    \begin{tabular}{|c|c|l|}
      \hline
      Версия & Дата релиза & Статус \\
      \hline
      22.0 & Июнь 2014 & Работает как ожидалось \\
      15.0 & Июль 2013 & Работает как ожидалось \\
      < 11.6 & Декабрь 2011 & Не поддерживается \\
      \hline
    \end{tabular}
  \end{center}
\end{table}

% написать как проводились вычисления fps
% различная визуализация
% n^2 и поиск ближайших
% сравнение по количеству частиц
% сравнение на разных gpu
% сравнение cpu и gpu
