\newpage
\section{Реализация}

\subsection{Технологии и библиотеки}

Как говорилось ранее, проект реализован под веб-платформу. Для разработки используются следующие технологии:

\begin{itemize}
  \item HTML5 -- язык разметки, используемый для построения структуры и представления веб-страниц. 
    Это пятая версия языка, которая добавляет поддержку тэга \textless{}canvas\textgreater{}, 
    который позволяет скриптовому попиксельному отображению изображений через различные контексты. 
    На момент написания работы, доступно два основных контекста: 2d и webgl.

    2D был первым реализованным типом контекста. Он реализован как абстрактный автомат (схоже 
    с OpenGL) и может быть использован для высокопроизводительной визуализации двухмерных объектов, 
    таких как линии, прямоугольники, кривые, bitmap-изображения и т.д.

    Следующий тип контекста позволил разработчикам создавать высокопроизводительные 3D изображения 
    без использования сторонних плагинов и расширений. Это значит что пользователям не требуется 
    устанавливать дополнительное ПО (например, Adobe Flash Player или Java VM) для просмотра.

  \item Javascript -- динамичный язык программирования. Используется для взаимодействия с 
    пользователями, контроля браузером и асинхронной загрузки ресурсов.

  \item Coffeescript -- язык программирования, который транслируется в javascript. Добавляет 
    синтаксический сахар для повышения краткости и читаемости кода. Например, реализация классов 
    (из объектно-ориентированного программирования), которые имеют четкую и понятную структуру.

  \item WebGL -- спецификация интерфейса для создания динамичных 2D и 3D сцен без использования 
    сторонних плагинов. Создан и поддерживается организацией Khoronos Group, на данный момент, 
    разрабатывающей спецификацию OpenGL.  WebGL служит связкой между высокоуровневым языком 
    JavaScript и низкоуровневыми операциями на графических процессорах.

  \item GLSL (OpenGL Shading Language) -- язык высокого уровня для программирования шейдеров. 
    Основным преимуществом GLSL является переносимость между платформами и ОС. Т.е. алгоритмы, 
    описанные в рамках данной работы, могут быть перенесены на другие платформы без изменения 
    кода программы.
\end{itemize}

Для упрощения разработки используется Zepto.js. Zepto.js -- библиотека для расширения функционала 
javascript. В частности реализует работу с AJAX запросами, которые используются для загрузки 
ресурсов. Является минималистичным аналогом jQuery, который также может являться альтернативой.

Основным преимуществом данного подхода к реализации является платформонезвисиммость и быстрота разработки.

\subsection{Общая архитектура}
\subsection{Описание алгоритмов}
\subsubsection{Сортировка}
\subsubsection{Поиск ближайших}
\subsubsection{Кластеризация}
\subsubsection{SPH}
