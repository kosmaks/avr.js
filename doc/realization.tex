\newpage
\section{Реализация}

\subsection{Технологии и библиотеки}

Как говорилось ранее, проект реализован под веб-платформу. Для разработки используются следующие технологии:

\begin{itemize}
  \item HTML5 -- язык разметки, используемый для построения структуры и представления веб-страниц. 
    Это пятая версия языка, которая добавляет поддержку тэга \textless{}canvas\textgreater{}, 
    который позволяет скриптовому попиксельному отображению изображений через различные контексты. 
    На момент написания работы, доступно два основных контекста: 2d и webgl.

    2D был первым реализованным типом контекста. Он реализован как абстрактный автомат (схоже 
    с OpenGL) и может быть использован для высокопроизводительной визуализации двухмерных объектов, 
    таких как линии, прямоугольники, кривые, bitmap-изображения и т.д.

    Следующий тип контекста позволил разработчикам создавать высокопроизводительные 3D изображения 
    без использования сторонних плагинов и расширений. Это значит что пользователям не требуется 
    устанавливать дополнительное ПО (например, Adobe Flash Player или Java VM) для просмотра.

  \item Javascript -- динамичный язык программирования. Используется для взаимодействия с 
    пользователями, контроля браузером и асинхронной загрузки ресурсов.

  \item Coffeescript -- язык программирования, который транслируется в javascript. Добавляет 
    синтаксический сахар для повышения краткости и читаемости кода. Например, классы 
    (из объектно-ориентированного программирования), которые имеют четкую и понятную структуру.

  \item WebGL -- спецификация интерфейса для создания динамичных 2D и 3D сцен без использования 
    сторонних плагинов. Создан и поддерживается организацией Khronos Group, на данный момент, 
    разрабатывающей спецификацию OpenGL. WebGL служит связкой между высокоуровневым языком 
    JavaScript и низкоуровневыми операциями на графических процессорах.

  \item GLSL (OpenGL Shading Language) -- язык высокого уровня для программирования шейдеров. 
    Основным преимуществом GLSL является переносимость между платформами и ОС. Т.е. алгоритмы, 
    описанные в рамках данной работы, могут быть перенесены на другие платформы без изменения 
    кода программы.
\end{itemize}

Для упрощения разработки используется Zepto.js. Zepto.js -- библиотека для расширения функционала 
javascript. В частности реализует работу с AJAX запросами, которые используются для загрузки 
ресурсов. Является минималистичным аналогом jQuery, который также может являться альтернативой.

Основным преимуществом данного подхода к реализации является платформонезвисиммость и быстрота 
разработки.

\subsection{Общая архитектура}
\subsection{Реализация демонстрационного примера}
\subsubsection{Гидродинамика сглаженных частиц}

Гидродинамика сглаженных частиц (англ. Smoothed Particle Hydrodynamics, SPH) -- вычислительный 
метод, используемый для симуляции флюидов.

Флюид -- текучие вещества. К ним можно отнести:
\begin{itemize}
  \item жидкости. Примером жидкости может служить вода, масло, и т.д.;
  \item газы. Плотность данной среды меньше чем жидкостей. Пример: воздух;
  \item плазма;
\end{itemize}

Симуляция флюидов обычно описывается уравнениями Навье-Стокса.
Они являются одними из важнейших в гидродинамике и применяются в математическом
моделировании природных явлений и технических задач.

\begin{equation}
\label{eq:nve1}
  \rho[\frac{\delta{}v}{\delta{}t} + v \bullet \bigtriangledown{}v] = \rho{}g - \bigtriangledown{}p + \mu{}\bigtriangledown^2v
\end{equation}

\begin{equation}
\label{eq:massCont}
  \rho(\bigtriangledown\bullet{}v) = 0
\end{equation}

Двигаясь за счет гравитации $g$, давления $\bigtriangledown{}p$ и скорости $\mu\bigtriangledown^2v$ частицы жидкости движутся из зоны высокого давления в зону низкого.

Еще один параметр который влияет на поведение флюидов -- вязкость $\mu$.
Примером текучего вещества с низкой вязкостью может являться вода, воздух.
С высокой: мед, грязь. \\

В данной работе рассмотрен упрощенный случай: симуляция несжимаемого потока
Ньютоновских флюидов. Уравнение \eqref{eq:nve1} описывает состояние жидкости.

Составляющие уравнения:

\begin{itemize}
  \item $\rho$ -- плотность (скалярная величина);
  \item $p$ -- давление (скалярная величина);
  \item $g$ -- вектор гравитации;
  \item $v$ -- вектор скорости;
\end{itemize}

Давление $p$ может быть представлено как

\begin{equation}
\label{eq:pressure}
  p = k(\rho - \rho_0)
\end{equation}

, где $\rho_0$ -- плотность вне флюида. \\

Уравнение \eqref{eq:massCont} может быть выполнено при условии что масса частиц
постоянна и частицы ниоткуда не создаются и никуда не исчезают.

Конечный вид уравнения для каждой частицы принимает вид:

\begin{equation}
\label{eq:}
\frac{dv_i}{dt} = g - \frac{1}{\rho_i}\bigtriangledown{}p + \frac{\mu}{\rho_i}\bigtriangledown^2v
\end{equation}

Суть вычислительного метода гидродинамики сглаженных частиц состоит в аппроксимации величин
уравнений Навье-Стокса с использованием функции сглаживающего ядра.

\begin{equation}
\label{eq:mon1992}
A_i(r) = \int{}A(r')W(r - r', h)dr' \approx \sum_{b}A(r_b)W(r - r_b, h)
\end{equation}

И аппроксимация в условиях уравнений Навье-Стокса:

\begin{equation}
\label{eq:}
  \rho_i \approx \sum_{j}m_jW(r - r_j, h)
\end{equation}

\begin{equation}
\label{eq:}
\frac{\bigtriangledown{}p_i}{\rho_i} \approx \sum_{j}m_j(\frac{p_i}{\rho_i^2} + \frac{p_j}{\rho_j^2})\bigtriangledown{}W(r - r_j, h)
\end{equation}

\begin{equation}
\label{eq:}
\frac{\mu}{\rho_i}\bigtriangledown^2v_i \approx \frac{\mu}{\rho_i}\sum_{j}m_j(\frac{v_j - v_i}{\rho_j})\bigtriangledown^2W(r - r_j, h)
\end{equation}

, где $m$ -- масса, $r$ -- положение, $h$ -- радиус взаимодействия частиц. \\

Ограничения, которые накладываются на функции сглаживающего ядра:

\begin{itemize}
  \item $W = 0$ когда $||r - r_j|| > h$
  \item $W$ суммируется в $1$ по сфере радиуса $h$
\end{itemize}

В данной работе используются следующие ядра:

\begin{equation}
\label{eq:}
W(r - r_j, h) \equiv \frac{315}{64\pi{}h^9}(h^2-||r-r_j||^2)^3
\end{equation}

\begin{equation}
\label{eq:}
\bigtriangledown{}W(r - r_j, h) \equiv \frac{-45}{\pi{}h^6}(h - ||r - r_b||)^2\frac{r - r_j}{||r - r_j||}
\end{equation}

\begin{equation}
\label{eq:}
\bigtriangledown^2W(r - r_j, h) \equiv \frac{45}{\pi{}h^6}(h - ||r - r_j||)
\end{equation}

Простой подход к реализации -- полный перебор. Преимуществом данного подходя является быстрота
реализации. Однако сложность данного алгоритма $O(n^2)$, что означает что на достаточно большом
количестве частиц, он будет работать не эффективно.

Методы оптимизации полного перебора:

\begin{itemize}
  \item Ввиду того, что значение функции ядер на расстоянии превышающем $h$ равны 0 -- нет смысла
    просчитывать взаимодействие между всеми частицами. Для этого необходимо реализовать
    алгоритм поиска ближайших частиц.
  \item Практика показывает, что при симуляции не обязательно учитывать все частицы в радиусе, а
    ограничить их максимальное количество. Данный порог может быть использован в качестве
    параметра, задаваемого пользователем при симуляции.
\end{itemize}

Данный вычислительный метод весьма чувствителен к значениям. Поэтому вычисления составляющих
уравнения будут производиться в масштабе $\times0.004$. \\

Детализация алгоритма:

\begin{enumerate}
  \item задаются начальные положения и скорости частиц;
  \item рассчитываются ближайшие частицы с помощью сетки;
  \item вычисляется значение плотности $\rho$ для каждой из частиц;
  \item вычисляется значения градиентов давления $\bigtriangledown{}p$ и делятся на плотность;
  \item вычисляется значение скоростей, зависимых от вязкости $\frac{\mu}{\rho_i}\bigtriangledown^2v$ ;
  \item полученные значения складываются, добавляется значение вектора гравитации. Полученный
    результат прибавляется к старому значению скорости;
  \item на основе скоростей частиц меняется их положение;
  \item переход на шаг 3;
\end{enumerate}

Каждый шаг алгоритма можно представить в виде шейдерной программы. Информация о каждой из частиц 
заносится в соответствующий кадровый буфер. 

Потребуются следующие структуры данных:

\begin{itemize}
  \item 2 буфера для значений положений (текущее и будущее состояние);
  \item 2 буфера для значений скоростей (текущее и будущее состояние);
  \item 1 буфер для значений плотности;
  \item 2 буфера для значений градиентов давления (текущее и будущее состояние);
  \item 2 буфера для значений скоростей от вязкости (текущее и будущее состояние);
  \item 2 буфера для значений скоростей от вязкости (текущее и будущее состояние);
  \item 2 буфера для значений сетки (текущее и будущее состояние);
  \item 27 буферов для хранения информации об индексах каждой из соседних ячеек сетки;
\end{itemize}

%используется в симуляции физики и видео играз
%что такое жидкости {
%  типы жидкостей: жидкости, газы, плазма
%  иногда пластик тоже, движется очень очень медленно
%}
%уравнения навьера-стока {
%  несжимаемый 
%  непрерывность массы
%  расписать состоявляющие и градиенты
%}
%метод для решения нсе {
%  рассказать про scale (0..100 однако выполняются вычисления на 0.004х
%  ядра [Monaghan 1992], апроксимация состявляющие нсе
%  ядра используемые в работе 
%  описание переменных алогоритма
%}
%реализация на glsl {
%  буферы
%  какие-нибудь особенности
%}
\subsubsection{Сортировка}
\subsubsection{Поиск ближайших}
