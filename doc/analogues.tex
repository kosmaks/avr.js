\newpage
\section{Обзор аналогов}

\subsection{Стандарты}

Помимо используемого в работе WebGL также разрабатывается стандарт WebCL, который описывает 
javascript-интерфейс стандарта OpenCL, т.е. API и расширения языка Си для организации
кросс-платформенных параллельных вычислений. Версия WebCL 1.0 была выпущена 19 марта 2014 года \cite{webcl10}.
Основным недостатком данного стандарта является отсутствие поддержки большинством браузеров.

\subsection{Библиотеки}
Существует множество библиотек для визуализации в веб-приложениях. Функционал который они реализуют:
изображение графиков, построение графов, визуализация трехмерных объектов.

Примеры:
\begin{itemize}
  \item Three.js -- кроссбраузерная библиотека JavaScript для создания анимированной 3D графики.
    Предоставляет полный доступ ко всем WebGL шейдерам. Может совместно использоваться с элементом
    HTML5 CANVAS, SVG или WebGL. Реализована работа с кадровыми буферами (framebuffer);

  \item Highcharts -- библиотека для построения интерактивных графиков в веб-проектах. Предоставляет
    богатый API для создания наглядных визуальных представлений;

  \item D3.js -- библиотека для визуализации данных. Позволяет изобразить данные в HTML формате. 
    К примеру, построить таблицу из массива чисел;
    
  \item Cytoscape.js -- библиотека для анализа и визуализации. Позволяет автоматически 
    строить различные типы графов.
\end{itemize}

Недостаток данных решений: они реализуют исключительно визуализацию данных, таким образом реализация
алгоритмов ложится на плечи программистов. Т.е. что они только частично решают поставленную
задачу.
