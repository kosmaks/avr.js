\newpage
\section{Обзор аналогов}

\subsection{Стандарты}

Помимо используемого в работе WebGL так же разрабатывается стандарт WebCL, который описывает 
javascript-интерфейс стандарта OpenCL, т.е. API и расширения языка Си для организации
кросс-платформенных параллельных вычислений. Версия WebCL 1.0 была выпущена 19 марта 2014 года.
Основным недостатком данного стандарта является отсутствие поддержки большинством браузеров.

\subsection{Библиотеки}
Существует множество библиотек для визуализации в веб-приложениях. Функционал который они реализуют:
\begin{itemize}
  \item Изображение графиков
  \item Построение графов
  \item Визуализация 3-х мерных объектов
\end{itemize}

Примеры:
\begin{itemize}
  \item Three.js -- кроссбраузерная библиотека JavaScript для создания анимированной 3D графики.
    Предоставляет полный доступ ко всем WebGL шейдерам. Может совместно использоваться с элементом
    HTML5 CANVAS, SVG или WebGL. Реализована работа с кадровыми буферами (framebuffer).

  \item D3.js -- библиотека для визуализации данных. Позволяет изобразить данные, представленные в
    формате языка на веб-странице. К примеру, построить таблицу из массива чисел.
    
  \item Cytoscape.js -- библиотека для анализа и визуализации. Позволяет автоматически 
    строить различные типы графов.

  \item Highcharts -- библиотека для построения интерактивных графиков в веб-проектах. Предоставляет
    богатый API для создания наглядных визуальных представлений.
\end{itemize}

Недостаток данных решений: они реализуют исключительно визуализацию данных, т.е. реализация
алгоритмов ложиться на плечи программистов. Отсюда следует что они только частично решают поставленную
задачу.
