\newpage
\section{Заключение}

В рамках данной работы изучены возможные способы визуализации в веб-браузерах,
найден механизм просчета физики большого количества частиц в реальном времени.
Разработан инструмент для создания интерактивных визуальных представлений.
Разработан пример визуализации вычислительного метода гидродинамики сглаженных 
частиц для демонстрации производительности и работы инструмента.

Проведены тесты производительности. Полученные данные доказывают что визуализация
выполняется достаточно быстро, чтобы сохранялась интерактивность.

Реализация отвечает требованиям, поставленным в техническом задании:

\begin{itemize}
  \item скорость визуализации $>25$ кадров в секунду при $10000$ частиц;
  \item разработан механизм взаимодействия с пользователем;
  \item программа исполняется в веб-браузере без использования дополнительных 
    плагинов и расширений;
\end{itemize}

Для достижения цели были поставлены и решены следующие задачи:

\begin{itemize}
  \item изучение возможностей графических процессоров для визуализации
    и вычисления физических данных;
  \item описан программный интерфейс работы с WebGL для создания
    визуальных представлений;
  \item разработка способов визуализации частиц;
  \item разработка методов взаимодействия пользователя с симуляцией;
  \item реализация показательного демонстрационного примера;
\end{itemize}

Пояснительная записка отражает описание всех этапов создания инструмента,
от постановки задачи и описания структуры до ее реализации.
