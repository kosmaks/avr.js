\newpage

\section{Внедрение} % 3

Физика большого количества частиц отлично подходит для симуляции множества
природных явлений. Поведение воды, горение огня, физика ткани и т.д.
Находится применение в астрофизике, химии и биологии. Кроме того,
данный подход к вычислениям подходит не только для визуализации, но 
и для вычислений введенных пользователем данных. Например, для кластеризации 
поисковой выдачи в географических информационных системах.

Благодаря выбранной платформе, задача донести информацию до конечного пользователя
значительно упрощается. Нет необходимости устанавливать дополнительные плагины
и расширения для веб-браузера. Упрощается внедрение в уже существующие системы, 
так как большинство обучающих и экспертных систем используют веб-интерфейс.

\subsection{Учебные курсы}

Все большую популярность набирают системы онлайн-образования. К ним можно
отнести такие проекты, как Coursera \footnote{http://coursera.org}, mooc.org \footnote{http://mooc.org}
разрабатываемый компанией Google. Данные системы образования распространяются бесплатно 
и работают как веб-приложения.

Так же подобные визуальные представления могут использоваться при публикации работ студентов 
и ученых на сайтах высших учебных заведений и исследовательских центров.

\subsection{Визуальные представления в экспертных системах}

Разрабатываемое решение может использоваться для создания визуальных представлений
в экспертных системах, предназначенных для диагностики, мониторинга и проектирования
в сложных процессах и механизмах. Полученные визуализации помогут максимально наглядно
донести информацию до пользователя.

\subsection{Игровая индустрия}

Существует несколько примеров игр, которые используют вычислительный метод, описанный 
в документе для создания игрового процесса. Это значит что разработанный инструмент
подходит для создания браузерных HTML5 игр.

Последние версии мобильных браузеров уже добавили поддержку включения WebGL в
экспериментальном режиме. Это значит что на мобильных устройствах так же можно будет
использовать подобный подход для создания визуальных эффектов.
