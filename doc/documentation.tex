\newpage
\section{Руководство пользователя}

Чтобы использовать инструмент, необходимо подключить внешний 
скрипт avr.js при помощи тэга SCRIPT.

%\begin{verbatim}
  %<script type='text/javascript' src='avr.js'></script>
%\end{verbatim}

Для инициализации используется конструктор класса AVR.Context. 
Первым параметром которому передается объект типа CANVAS.

%\begin{verbatim}
  %var avr = new AVR.Context(document.getElementById('display'));
%\end{verbatim}

Для создания визуализации используется метод loadPrograms. Он
позволяет загрузить нужные шейдеры и передать константы. Каждый
из шейдеров автоматически проходит предварительную обработку.
При необходимости в консоль браузера выведется ошибка компиляции.

%\begin{verbatim}
  %avr.loadPrograms({
    %// описание и ссылки на нужные программы
  %}, {
    %// константы
  %}, function(prog) {
    %// код визуализации
  %});
%\end{verbatim}

По-умолчанию для каждой шейдерной программы необходимо указать
vertexUrl и fragmentUrl для вершинного и фрагментного шейдеров соответственно.
Если фрагментный шейдер предполагается использовать для просчета частиц,
ссылку на вершинный шейдер можно опустить.

%\begin{verbatim}
  %avr.loadPrograms({
    %// ...
    %genGridCells : "shaders/gen_grid_cells.glsl",
    %genAccessor  : "shaders/gen_accessor.glsl",
    %bitonicSort  : "shaders/bitonic.glsl",
    %// ..
   
    %axes: {
      %vertexUrl: "shaders/axes.vertex.glsl",
      %fragmentUrl: "shaders/axes.fragment.glsl",
    %}
  %}, { ...
%\end{verbatim}

После загрузки, программы будут доступны в функции обратного вызова.

Чтобы создать конвейер для обработки частиц можно либо вручную писать
использование программ с помощью метода use, либо использовать встроенный
компонент AVR.Chain. Создать цепь можно с помощью метода createChain.

Для того чтобы исполнить программу и вывести содержимое на экран нужно
вызывать метод pass, первым параметром к которому будет программа типа
AVR.Program.

%\begin{verbatim}
  %var c = avr.createChain();
  %c.pass(prog.helloWorld);
%\end{verbatim}

В данном случае начальными данными для вершинного шейдера будут 4
вершины, расположенные в углах экрана (от -1 до 1). Размер кадрового
буфера определяется размером элемента.

Для создания фрагментного буфера в цепи необходимо использовать метод
framebuffer. Первый параметр -- название буфер, второй -- информация о 
фрагментном буфере: размер (size), формат (format), тип (type), начальные
данные (data). Для создания двойного буфера используется метод doubleFramebuffer.
Параметры те же.

%\begin{verbatim}
  %c.framebuffer('reader', { size: size });
  %c.doubleFramebuffer('grid', { size: size });
  %c.doubleFramebuffer('particles', { size: size });
%\end{verbatim}

Чтобы указать в какой кадровый буфер записать результат программы, нужно 
указать селектор вторым параметром в методе pass.

%\begin{verbatim}
  %c.pass(p.fill, 'back particles');
  %c.pass(p.zero, 'back velocities');
  %c.pass(p.zero, 'auto densities');
  %c.pass(p.zero, 'auto pressures');
%\end{verbatim}

Для того, чтобы передать содержимое других буферов в программу третьим
параметром в pass указывается объект где ключами являются названия uniform
переменных, а значениями селектор кадрового буфера.

%\begin{verbatim}
  %c.pass(p.position, 'front particles', {
    %back: 'back particles',
    %velocities: 'front velocities'
  %});
%\end{verbatim}

Для более детальной настройки данного этапа вычислений можно использовать
функцию обратного вызова в методе pass. Первый параметр которой -- объект
содержащий кадровый буфер и программу типов AVR.Framebuffer и AVR.Program.

%\begin{verbatim}
  %c.pass(p.velocity, 'front velocities', {
    %back: 'back velocities',
    %particles: 'back particles',
    %pressures: 'auto pressures',
    %viscosity: 'auto viscosity'
  %}, function(prog) {
    %prog.sendFloat('wallDispl', wallDispl)
  %});
%\end{verbatim}

Для сортировки данных в кадровом буфере используется алгоритмом
битонической сортировки, необходимо использовать метод bitonicSort,
первым параметром которого является название двойного кадрового буфера.
Сортировка происходит по четвертой координате $w$.

%\begin{verbatim}
  %c.bitonicSort('grid');
%\end{verbatim}

Для того чтобы визуализировать результат вычислений, необходимо создать
экземпляр класса AVR.Visual при помощи метода createVisual. Затем
при помощи метода visualize создавать визуальное представление.
Первым параметром в visualize является тип визуализации. Вторым параметром
передается объект для конфигурации визуального представления.

Реализованы следующие типы визуального представления:

\begin{itemize}
  \item AVR.Axes -- рисует оси координат;
  \item AVR.Points -- изобразить в виде точек. По ключу particles
    указывается кадровый буфер из которого будут браться координаты частиц.
    По ключу colors указывается буфер для получения цветов каждой из частиц;
  \item AVR.Vectors -- изобразить в виде векторов. Ключ particles
    указывает кадровый буфер для координат частиц. Ключ colors
    принимает массив из двух элементов с цветами начала и конца вектора. Ключ
    vectors указывает на буфер векторов. Ключ scale содержит число масштаба
    векторов.
\end{itemize}

Для очистки экрана используется метод clear. Перед генерацией следующего
кадра вызывается метод swapBuffers, который меняет местами двойные буферы.

%\begin{verbatim}
  %avr.clear();

  %vis.visualize(AVR.Axes);

  %vis.visualize(AVR.Particles, {
    %positions: c.getBuffer('front particles')
  %});

  %vis.visualize(AVR.Vectors, {
    %positions: c.getBuffer('front particles'),
    %vectors: c.getBuffer('front velocities'),
    %scale: 7
  %});

  %c.swapBuffers();
%\end{verbatim}

Для того чтобы использовать константы в шейдерах используются синтаксис
\$названиеКонстанты.

%\begin{verbatim}
  %vec2 cursor = floor(vec2($sizex, $sizey) / 2.);
%\end{verbatim}

Чтобы исключить дублирование кода в шейдерах, реализовано включение
кода одного из шейдеров в код текущего при помощи директивы. При этом
шейдеры загружаются один раз и включаются в месте написания директивы.

%\begin{verbatim}
  %attribute vec3 vertex;
  %uniform sampler2D positions;

  %$include "shaders/transform.glsl"
 
  %void main() {
    %// ...
%\end{verbatim}

Для создания перспективы и работы с пользовательским вводом
реализованы основные функции в шейдере ``shader/transform.glsl''.

%\begin{verbatim}
  %vec4 perspective(vec3 src); // создает перспективу для точки
  %bool mouseTouch(vec3 part); // возвращает true, если данная 
                                 %частица попала в радиус нажатия

  %vec3 rotation; // содержит текущие углы вращения
  %vec2 mouse; // содержит текущую координату мыши в мировых
                 %координатах (от -1 до 1)
%\end{verbatim}
