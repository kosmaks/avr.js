\documentclass[a4paper,14pt,russian,nocolumnsxix,nocolumnxxxii,nocolumnxxxi,hpadding=10mm]{eskdtext}

\usepackage{cmap}
\usepackage[T2A]{fontenc}
\usepackage[utf8x]{inputenc}
\usepackage[russian]{babel}
\usepackage{lastpage}
\usepackage{graphicx}
\usepackage{longtable}
\usepackage{amsmath}
\usepackage{hyperref}
\usepackage{glossaries}

\renewcommand{\theenumi}{\arabic{enumi}} % Меняем везде перечисления на цифра.цифра
\renewcommand{\labelenumi}{\arabic{enumi}} % Меняем везде перечисления на цифра.цифра
\renewcommand{\theenumii}{.\arabic{enumii}} % Меняем везде перечисления на цифра.цифра
\renewcommand{\labelenumii}{\arabic{enumi}.\arabic{enumii}.} % Меняем везде перечисления на цифра.цифра
\renewcommand{\theenumiii}{.\arabic{enumiii}} % Меняем везде перечисления на цифра.цифра
\renewcommand{\labelenumiii}{\arabic{enumi}.\arabic{enumii}.\arabic{enumiii}.} % Меняем везде перечисления на цифра.цифра

\newcommand{\nocontentsline}[3]{}
\newcommand{\tocless}[2]{\bgroup\let\addcontentsline=\nocontentsline#1{#2}\egroup}

% Font settings
\linespread{1.3} % 1.5
%\renewcommand{\rmdefault}{ftm} % Times new roman
\frenchspacing

%\ESKDsectStyle{section}{\normalsize} % Заголовки глав обычным шрифтом
\ESKDsectStyle{subsection}{\normalsize} % Заголовки разделов обычным шрифтом
\ESKDsectStyle{subsubsection}{\normalsize} % Заголовки подразделов обычным шрифтом

\begin{titlepage}
\newpage

\begin{center}
  ФЕДЕРАЛЬНОЕ ГОСУДАРСТВЕННОЕ БЮДЖЕТНОЕ ОБРАЗОВАТЕЛЬНОЕ УЧРЕЖДЕНИЕ \\*
  ВЫСШЕГО ПРОФЕССИОНАЛЬНОГО ОБРАЗОВАНИЯ \\*
  ``Южно-Уральский Государственный Университет''
\end{center}

\vspace{8em}

\begin{center}
  \large Пояснительная записка \\ к дипломному проекту на тему: \\*
\end{center}

\vspace{2.5em}

\begin{center}
  \large ``Алгоритмизация визуальных представлений''
\end{center}

\vspace{6em}

\begin{flushleft}
  Студент--дипломник \hrulefill Костюченко М.А. \\
  \vspace{1.5em}
  Научный руководитель \\
  доцент \hrulefill Кафтанников И.Л. \\
  \vspace{1.5em}
  Рецензент \\
  \\
  \vspace{1.5em}
\end{flushleft}

\vspace{\fill}

\begin{center}
  Челябинск, 2014.
\end{center}

\end{titlepage}


\begin{document}
  \maketitle
  \newpage
\section*{Аннотация}

\hfill
\begin{minipage}[t]{0.62\textwidth}
  \ESKDtheAuthor \space \ESKDtheDocName -- Челябинск ЮУрГУ, ПС, 2013. -- \ESKDtotal{page} с., 10~ил., библиогр. список~--~\ESKDtotal{bibitem}
\end{minipage} \\
\\

В данной работе рассмотрена разработка инструмента для создания визуальных
представлений, требующих параллельных вычислений на большом объеме данных, 
в веб-приложениях.

Целью дипломной работы явилось разработка программного инструмента для эффективного
создания визуальных представлений в веб-приложениях. Рассмотренные типы
визуализаций требуют особого подхода к вычислениям, так как отображают
взаимодействие большого количества частиц (от десяти тысяч) в реальном времени.
Подобный подход к вычислениям может быть использован для симуляции природных явлений 
и вычислении операций на большом количестве данных. Для достижения цели были поставлены
задачи: изучить способы визуализации данных в веб-приложениях с целью поиска эффективного
способа отображения, создание инструмента для визуализации и реализация демонстрационного 
примера для тестирования производительности.

Ожидаемые результаты:

\begin{itemize}
  \item эффективный способ визуализации сложных процессов в \\веб-приложениях;
  \item уменьшение времени на разработку подобных визуальных представлений.
\end{itemize}

Область применения -- обучающие сервисы, научные статьи и экспертные системы.

  \newpage
  \tableofcontents 
  \newpage
\section*{Введение}

%Проблема: есть некоторый описанный процесс . Задача сделать интерактивную визуализацию большого количества взаимодействующих объектов.
%Критерии: интерактивность -> реал-тайм, должно выполняться в браузере.
%Вопрос как выполнить это максимально рационально.
%Решение: Реализация основных алгоритмов, которые могут понадобиться при создании визуализации.
%Пример использования: визуализация пользовательских данных различных для каждого пользователя. При создании учебных курсов.  Создание визуализаций в экспертных системах. Кластеризация поисковой выдачи в ГИС.

%Графическое представление заметно способствует усваиванию информации.
%Современные компьютерные технологии помогают разрабатывать интерактивные и 
%наглядные изображения информации. Интерактивная визуализация информация -- подход
%к обработке информации в информационных системах, которая превращается в непрерывный 
%процесс взаимодействия с информацией через визуальное отображение.
%Приложения визуализации информации возникают в таких областях, как информационные
%системы и программное обеспечение, биологические науки, искусственный интеллект и 
%анализ пользовательских данных.

Визуальное представление информации -- это интерпретация числовой и текстовой
информации в виде графиков, структурных схем, таблиц, карт и т.д.
.... еще какая-нибудь хуита про интерактивную визуализацию
\linebreak \\
Несмотря на развитие современных подходов к интерактивному отображению, задача
визуализации систем, требующих оперативного взаимодействие большого количества
объектов, все еще остается не тривиальной. Критериями эффективности решения
являются качество и доступность итоговой визуализации, а также время, необходимое
для ее создание.
\linebreak \\
Развитие параллельных вычислений с использованием графических процессоров сделало 
возможным визуализацию сложных интерактивных представлений информации. 
\linebreak \\
Веб приложения являются удачным решением быстрого донесения информации до большого
количества пользователей. Поддержка браузерами Новых стандарты такие как HTML5 и WebGL 
позволила использовать преимущества вычислительных мощностей GPU в веб-приложениях.
\linebreak \\
В данной работе предлагается решение для упрощения создания визуальных представлений
в веб-приложениях. Идея состоит в реализации алгоритмов и написании упрощенного 
API для использования WebGL в качестве инструмента параллельных вычислений.
В качестве демонстрации использования данного решения реализована визуализация
вычислительного метода для симуляции жидкостей и газов -- гидродинамика сглаженных частиц.

%Легче воспринимать информацию с интерактивными примерами.
%Стоит задача как визуализировать процессы где требуется много вычислений.
%Современные компьютерная графика и постоянное развитие.
%Современные веб-технологии, кросс-платформенность и прочие плюхи веба.
%Использование webgl как инструмента для вычисления и визуализации.

 % 1 -> 2
  \newpage
\section{ГЛОССАРИЙ}

\begin{itemize}
  \item Визуальное представление -- интерпретация числовой и текстовой информации в 
    виде графиков, структурных схем, таблиц, карт и изображений.
  \item ГП (или GPU) -- графический процессор.
  \item ЦП -- центральный процессор.
  \item Шейдер -- программа, исполняемая на графическом процессоре.
  \item Фрагмент -- участок размером с пиксель.
  \item Проход (в графическом конвейере) -- проход данных по графическому конвейеру.
  \item Селектор (кадрового буфера) -- параметр, указывающий какой из двух кадровых буферов
    необходимо вернуть.
  \item Вершинный буфер -- особенность OpenGL, обеспечивающая загрузку данных на видеоустройство.
  \item Кадровый буфер -- область памяти видеокарты для хранения изображения.
  \item Частица -- термин, который используется для обозначения объектов, которые в 
    контексте физической симуляции можно считать неделимыми.
  \item SPH (англ. Smoothed Particle Hydronynamics) -- гидродинамика сглаженных частиц. 
    Вычислительный метод для симуляции жидкостей и газов.
  \item UV-координата -- двумерный вектор, обозначающий положение на текстуре.
\end{itemize}

  \newpage
\section{Техническое задание}

Необходимо разработать инструмент для вычисления и создания визуальных 
представлений большого количества оперативно взаимодействующих частиц.

Разработать визуализацию симуляции жидкости методом гидродинамики сглаженных
частиц в целях демонстрации работы инструмента.

Программное обеспечение должно включать в себя следующее:

\begin{itemize}
  \item API для работы с массивами частиц и их характеристиками;
  \item Типы визуальных представлений:
    \begin{itemize}
      \item Представление точек с заданными характеристиками в трехмерном пространстве;
      \item Представление векторов в трехмерном пространстве;
      %\item Реконструкция поверхности методом шагающих кубиков;
    \end{itemize}
  \item Способы взаимодействия пользователя с симуляцией:
    \begin{itemize}
      \item Воздействие на частицы методом отбрасывания лучей;
      \item Смена угла обзора;
    \end{itemize}
\end{itemize}

Требования к ПО и демонстрационному примеру:

\begin{itemize}
  \item Должно выполняться в веб-браузере;
  \item Полученная визуализация должна быть интерактивной;
  \item Симуляция должна генерировать 25-60 кадров в секунд 
    при количестве частиц свыше 10000;
\end{itemize}

\subsection{Этап первый}

На данном этапе необходимо провести сравнительный анализ возможных способов визуализации
трехмерных изображений в веб-браузере. Посмотреть ситуации, когда необходим данный тип 
визуальных представлений, чтобы сделать максимально универсальный программный интерфейс.

Следующий щаг -- обзор способов вычисления таких объемов данных. Поиск оптимальных
алгоритмов для реализации демонстрационного примера.

Затем необходимо составить программный интерфейс исходя из собранных сведений.

Когда вся информация собрана, следует переходить к следующему этапу.

\subsection{Этап второй}

На этом этапе выполняется реализация технической части работы. Написание программного
интерфейса и его реализация. 

Далее разрабатываются перечисленные выше типы визуальных представлений. Реализуются
способы взаимодействия пользователя с симуляцией.

Третий шаг -- реализация алгоритмов демонстрационного примера.

После реализации и тестирования необходимо разработать инструкцию по использованию, а
также задокументировать исходный код.

 % 2
  \newpage
\section{Обзор аналогов}

\subsection{Стандарты}

Помимо используемого в работе WebGL так же разрабатывается стандарт WebCL, который описывает 
javascript-интерфейс стандарта OpenCL, т.е. API и расширения языка Си для организации
кросс-платформенных параллельных вычислений. Версия WebCL 1.0 была выпущена 19 марта 2014 года.
Основным недостатком данного стандарта является отсутствие поддержки большинством браузеров.

\subsection{Библиотеки}
Существует множество библиотек для визуализации в веб-приложениях. Функционал который они реализуют:
\begin{itemize}
  \item Изображение графиков
  \item Построение графов
  \item Визуализация 3-х мерных объектов
\end{itemize}

Примеры:
\begin{itemize}
  \item Three.js -- кроссбраузерная библиотека JavaScript для создания анимированной 3D графики.
    Предоставляет полный доступ ко всем WebGL шейдерам. Может совместно использоваться с элементом
    HTML5 CANVAS, SVG или WebGL. Реализована работа с кадровыми буферами (framebuffer).

  \item D3.js -- библиотека для визуализации данных. Позволяет изобразить данные, представленные в
    формате языка на веб-странице. К примеру, построить таблицу из массива чисел.
    
  \item Cytoscape.js -- библиотека для анализа и визуализации. Позволяет автоматически 
    строить различные типы графов.

  \item Highcharts -- библиотека для построения интерактивных графиков в веб-проектах. Предоставляет
    богатый API для создания наглядных визуальных представлений.
\end{itemize}

Недостаток данных решений: они реализуют исключительно визуализацию данных, т.е. реализация
алгоритмов ложиться на плечи программистов. Отсюда следует что они только частично решают поставленную
задачу.
 % 2
  \newpage
\section{Особенности вычислений на GPU}
 % 4 -> 5?? (pictures)
  \newpage
\section{Реализация}

\subsection{Технологии и библиотеки}

Как говорилось ранее, проект реализован под веб-платформу. Для разработки используются следующие технологии:

\begin{itemize}
  \item HTML5 -- язык разметки, используемый для построения структуры и представления веб-страниц. 
    Это пятая версия языка, которая добавляет поддержку тэга \textless{}canvas\textgreater{}, 
    который позволяет скриптовому попиксельному отображению изображений через различные контексты. 
    На момент написания работы, доступно два основных контекста: 2d и webgl.

    2D был первым реализованным типом контекста. Он реализован как абстрактный автомат (схоже 
    с OpenGL) и может быть использован для высокопроизводительной визуализации двухмерных объектов, 
    таких как линии, прямоугольники, кривые, bitmap-изображения и т.д.

    Следующий тип контекста позволил разработчикам создавать высокопроизводительные 3D изображения 
    без использования сторонних плагинов и расширений. Это значит что пользователям не требуется 
    устанавливать дополнительное ПО (например, Adobe Flash Player или Java VM) для просмотра.

  \item Javascript -- динамичный язык программирования. Используется для взаимодействия с 
    пользователями, контроля браузером и асинхронной загрузки ресурсов.

  \item Coffeescript -- язык программирования, который транслируется в javascript. Добавляет 
    синтаксический сахар для повышения краткости и читаемости кода. Например, классы 
    (из объектно-ориентированного программирования), которые имеют четкую и понятную структуру.

  \item WebGL -- спецификация интерфейса для создания динамичных 2D и 3D сцен без использования 
    сторонних плагинов. Создан и поддерживается организацией Khronos Group, на данный момент, 
    разрабатывающей спецификацию OpenGL. WebGL служит связкой между высокоуровневым языком 
    JavaScript и низкоуровневыми операциями на графических процессорах.

  \item GLSL (OpenGL Shading Language) -- язык высокого уровня для программирования шейдеров. 
    Основным преимуществом GLSL является переносимость между платформами и ОС. Т.е. алгоритмы, 
    описанные в рамках данной работы, могут быть перенесены на другие платформы без изменения 
    кода программы.
\end{itemize}

Для упрощения разработки используется Zepto.js. Zepto.js -- библиотека для расширения функционала 
javascript. В частности реализует работу с AJAX запросами, которые используются для загрузки 
ресурсов. Является минималистичным аналогом jQuery, который также может являться альтернативой.

Основным преимуществом данного подхода к реализации является платформонезвисиммость и быстрота 
разработки.

\subsection{Общая архитектура}
\subsection{Реализация демонстрационного примера}
\subsubsection{Гидродинамика сглаженных частиц}

Гидродинамика сглаженных частиц (англ. Smoothed Particle Hydrodynamics, SPH) -- вычислительный 
метод, используемый для симуляции флюидов.

Флюид -- текучие вещества. К ним можно отнести:
\begin{itemize}
  \item жидкости. Примером жидкости может служить вода, масло, и т.д.;
  \item газы. Плотность данной среды меньше чем жидкостей. Пример: воздух;
  \item плазма;
\end{itemize}

Симуляция флюидов обычно описывается уравнениями Навье-Стокса.
Они являются одними из важнейших в гидродинамике и применяются в математическом
моделировании природных явлений и технических задач.

\begin{equation}
\label{eq:nve1}
  \rho[\frac{\delta{}v}{\delta{}t} + v \bullet \bigtriangledown{}v] = \rho{}g - \bigtriangledown{}p + \mu{}\bigtriangledown^2v
\end{equation}

\begin{equation}
\label{eq:massCont}
  \rho(\bigtriangledown\bullet{}v) = 0
\end{equation}

Двигаясь за счет гравитации $g$, давления $\bigtriangledown{}p$ и скорости $\mu\bigtriangledown^2v$ частицы жидкости движутся из зоны высокого давления в зону низкого.

Еще один параметр который влияет на поведение флюидов -- вязкость $\mu$.
Примером текучего вещества с низкой вязкостью может являться вода, воздух.
С высокой: мед, грязь. \\

В данной работе рассмотрен упрощенный случай: симуляция несжимаемого потока
Ньютоновских флюидов. Уравнение \eqref{eq:nve1} описывает состояние жидкости.

Составляющие уравнения:

\begin{itemize}
  \item $\rho$ -- плотность (скалярная величина);
  \item $p$ -- давление (скалярная величина);
  \item $g$ -- вектор гравитации;
  \item $v$ -- вектор скорости;
\end{itemize}

Давление $p$ может быть представлено как

\begin{equation}
\label{eq:pressure}
  p = k(\rho - \rho_0)
\end{equation}

, где $\rho_0$ -- плотность вне флюида. \\

Уравнение \eqref{eq:massCont} может быть выполнено при условии что масса частиц
постоянна и частицы ниоткуда не создаются и никуда не исчезают.

Конечный вид уравнения для каждой частицы принимает вид:

\begin{equation}
\label{eq:}
\frac{dv_i}{dt} = g - \frac{1}{\rho_i}\bigtriangledown{}p + \frac{\mu}{\rho_i}\bigtriangledown^2v
\end{equation}

Суть вычислительного метода гидродинамики сглаженных частиц состоит в аппроксимации величин
уравнений Навье-Стокса с использованием функции сглаживающего ядра.

\begin{equation}
\label{eq:mon1992}
A_i(r) = \int{}A(r')W(r - r', h)dr' \approx \sum_{b}A(r_b)W(r - r_b, h)
\end{equation}

И аппроксимация в условиях уравнений Навье-Стокса:

\begin{equation}
\label{eq:}
  \rho_i \approx \sum_{j}m_jW(r - r_j, h)
\end{equation}

\begin{equation}
\label{eq:}
\frac{\bigtriangledown{}p_i}{\rho_i} \approx \sum_{j}m_j(\frac{p_i}{\rho_i^2} + \frac{p_j}{\rho_j^2})\bigtriangledown{}W(r - r_j, h)
\end{equation}

\begin{equation}
\label{eq:}
\frac{\mu}{\rho_i}\bigtriangledown^2v_i \approx \frac{\mu}{\rho_i}\sum_{j}m_j(\frac{v_j - v_i}{\rho_j})\bigtriangledown^2W(r - r_j, h)
\end{equation}

, где $m$ -- масса, $r$ -- положение, $h$ -- радиус взаимодействия частиц. \\

Ограничения, которые накладываются на функции сглаживающего ядра:

\begin{itemize}
  \item $W = 0$ когда $||r - r_j|| > h$
  \item $W$ суммируется в $1$ по сфере радиуса $h$
\end{itemize}

В данной работе используются следующие ядра:

\begin{equation}
\label{eq:}
W(r - r_j, h) \equiv \frac{315}{64\pi{}h^9}(h^2-||r-r_j||^2)^3
\end{equation}

\begin{equation}
\label{eq:}
\bigtriangledown{}W(r - r_j, h) \equiv \frac{-45}{\pi{}h^6}(h - ||r - r_b||)^2\frac{r - r_j}{||r - r_j||}
\end{equation}

\begin{equation}
\label{eq:}
\bigtriangledown^2W(r - r_j, h) \equiv \frac{45}{\pi{}h^6}(h - ||r - r_j||)
\end{equation}

Простой подход к реализации -- полный перебор. Преимуществом данного подходя является быстрота
реализации. Однако сложность данного алгоритма $O(n^2)$, что означает что на достаточно большом
количестве частиц, он будет работать не эффективно.

Методы оптимизации полного перебора:

\begin{itemize}
  \item Ввиду того, что значение функции ядер на расстоянии превышающем $h$ равны 0 -- нет смысла
    просчитывать взаимодействие между всеми частицами. Для этого необходимо реализовать
    алгоритм поиска ближайших частиц.
  \item Практика показывает, что при симуляции не обязательно учитывать все частицы в радиусе, а
    ограничить их максимальное количество. Данный порог может быть использован в качестве
    параметра, задаваемого пользователем при симуляции.
\end{itemize}

Данный вычислительный метод весьма чувствителен к значениям. Поэтому вычисления составляющих
уравнения будут производиться в масштабе $\times0.004$. \\

Детализация алгоритма:

\begin{enumerate}
  \item задаются начальные положения и скорости частиц;
  \item рассчитываются ближайшие частицы с помощью сетки;
  \item вычисляется значение плотности $\rho$ для каждой из частиц;
  \item вычисляется значения градиентов давления $\bigtriangledown{}p$ и делятся на плотность;
  \item вычисляется значение скоростей, зависимых от вязкости $\frac{\mu}{\rho_i}\bigtriangledown^2v$ ;
  \item полученные значения складываются, добавляется значение вектора гравитации. Полученный
    результат прибавляется к старому значению скорости;
  \item на основе скоростей частиц меняется их положение;
  \item переход на шаг 3;
\end{enumerate}

Каждый шаг алгоритма можно представить в виде шейдерной программы. Информация о каждой из частиц 
заносится в соответствующий кадровый буфер. 

Потребуются следующие структуры данных:

\begin{itemize}
  \item 2 буфера для значений положений (текущее и будущее состояние);
  \item 2 буфера для значений скоростей (текущее и будущее состояние);
  \item 1 буфер для значений плотности;
  \item 2 буфера для значений градиентов давления (текущее и будущее состояние);
  \item 2 буфера для значений скоростей от вязкости (текущее и будущее состояние);
  \item 2 буфера для значений скоростей от вязкости (текущее и будущее состояние);
  \item 2 буфера для значений сетки (текущее и будущее состояние);
  \item 27 буферов для хранения информации об индексах каждой из соседних ячеек сетки;
\end{itemize}

%используется в симуляции физики и видео играз
%что такое жидкости {
%  типы жидкостей: жидкости, газы, плазма
%  иногда пластик тоже, движется очень очень медленно
%}
%уравнения навьера-стока {
%  несжимаемый 
%  непрерывность массы
%  расписать состоявляющие и градиенты
%}
%метод для решения нсе {
%  рассказать про scale (0..100 однако выполняются вычисления на 0.004х
%  ядра [Monaghan 1992], апроксимация состявляющие нсе
%  ядра используемые в работе 
%  описание переменных алогоритма
%}
%реализация на glsl {
%  буферы
%  какие-нибудь особенности
%}
\subsubsection{Сортировка}
\subsubsection{Поиск ближайших}
 % 7
  \newpage
\section{Тестирование производительности}

Все тесты проводились на демонстрационном примере с различной конфигурацией
симуляции. Основной характеристикой производительности является количество 
кадров в секунду, генерируемых визуализацией.

Для вычисления количества кадров в секунду используются функции стандартной 
библиотеки javascript. Данные снимаются по минимальным значениям показателей.
Для чистоты эксперимента сравнения проводились на разных компьютерах. \\

Конфигурация первого компьютера:

\begin{itemize}
  \item Видеокарта -- Intel HD 4000;
  \item Процессор -- Intel Core i5;
  \item Объем ОЗУ -- 4 гб;
\end{itemize}

Результаты вычислений с различными типами визуализации приведены 
в таблицах \ref{tab:fst:simple}. По горизонтали изменяется количество частиц.
По вертикали тип визуализации.

\begin{table}[H]
  \caption{\label{tab:fst:simple}Комп. №1. Зависимость от типа визуализации}
  \begin{center}
    \begin{tabular}{|c|c|c|c|c|c|}
      \hline
      Тип визуализации & 100 ч. & 1000 ч. & 5000 ч. & 10000 ч. & 100000 ч. \\
      \hline
      Облако точек & 320 & 175 & 97 & 40 & 24 \\
      Направленные векторы & 320 & 170 & 95 & 39 & 23 \\
      \hline
    \end{tabular}
  \end{center}
\end{table}

Одним из параметров, который существенно влияет на скорость визуализации -- это
радиус взаимодействия частиц. Чем он больше, тем большее количество частиц необходимо
обработать. Размер пространства $150\times150$. Радиус указывается в измерениях
виртуального пространства. Результаты тестирования приведены в таблице \ref{tab:fst:radius}.

\begin{table}[H]
  \caption{\label{tab:fst:radius}Комп. №1. Зависимость от радиуса взаимодействия}
  \begin{center}
    \begin{tabular}{|c|c|c|c|c|c|}
      \hline
      Радиус & 100 ч. & 1000 ч. & 5000 ч. & 10000 ч. & 100000 ч. \\
      \hline
      5 & 320 & 175 & 97 & 40 & 24 \\
      10 & 250 & 120 & 84 & 37 & 22 \\
      50 & 100 & 90 & 43 & 20 & 8 \\
      \hline
    \end{tabular}
  \end{center}
\end{table}

Для демонстрации производительности алгоритма поиска ближайших, проведены
тесты с полным перебором. Результаты приведены в таблице \ref{tab:fst:algorithm}.

\begin{table}[H]
  \caption{\label{tab:fst:algorithm}Комп. №1. Зависимость от алгоритма поиска ближайших частиц}
  \begin{center}
    \begin{tabular}{|c|c|c|c|c|c|}
      \hline
      Алгоритм & 100 ч. & 1000 ч. & 5000 ч. & 10000 ч. & 100000 ч. \\
      \hline
      Использование сетки & 320 & 175 & 97 & 40 & 24 \\
      Полный перебор & 343 & 170 & 70 & 20 & 2 \\
      \hline
    \end{tabular}
  \end{center}
\end{table}

В целях демонстрации, так же разработан просчет визуализации на центральном процессоре.
При этом время, затраченное на отображение, не учитывается. Результаты тестирования 
приведены в таблице \ref{tab:fst:cpu}.  \\

\begin{table}[H] 
  \caption{\label{tab:fst:cpu}Комп. №1. Зависимость от типа процессора} 
  \begin{center} 
    \begin{tabular}{|c|c|c|c|c|c|} 
      \hline
      Тип процессора & 100 ч. & 1000 ч. & 5000 ч. & 10000 ч. & 100000 ч. \\
      \hline
      Графический & 320 & 175 & 97 & 40 & 24 \\
      Центральный & 140 & 100 & 30 & 7 & $\approx{}0.3$ \\
      \hline
    \end{tabular} 
  \end{center} 
\end{table} 

Конфигурация второго компьютера:

\begin{itemize} 
  \item Видеокарта -- Nvidia GeForce 640;
  \item Процессор -- Intel Core i7;
  \item Объем ОЗУ -- 8 гб;
\end{itemize} 

Для второго компьютера проведены те же тесты. Результаты приведены в таблицах
\ref{tab:snd:simple}, \ref{tab:snd:radius}, \ref{tab:snd:algorithm}, \ref{tab:snd:cpu}.

\begin{table}[H] 
  \caption{\label{tab:snd:simple}Комп. №2. Зависимость от типа визуализации} 
  \begin{center}
    \begin{tabular}{|c|c|c|c|c|c|}
      \hline
      Тип визуализации & 100 ч. & 1000 ч. & 5000 ч. & 10000 ч. & 100000 ч. \\
      \hline
      Облако точек & 398 & 315 & 150 & 64 & 47 \\
      Направленные векторы & 399 & 310 & 152 & 63 & 46 \\
      \hline
    \end{tabular}
  \end{center}
\end{table}

\begin{table}[H]
  \caption{\label{tab:snd:radius}Комп. №2. Зависимость от радиуса взаимодействия}
  \begin{center}
    \begin{tabular}{|c|c|c|c|c|c|}
      \hline
      Радиус & 100 ч. & 1000 ч. & 5000 ч. & 10000 ч. & 100000 ч. \\
      \hline
      5 & 398 & 315 & 150 & 64 & 47 \\
      10 & 330 & 205 & 130 & 37 & 22 \\
      50 & 340 & 110 & 78 & 20 & 11 \\
      \hline
    \end{tabular}
  \end{center}
\end{table}

\begin{table}[H]
  \caption{\label{tab:snd:algorithm}Комп. №3. Зависимость от алгоритма поиска ближайших частиц}
  \begin{center}
    \begin{tabular}{|c|c|c|c|c|c|}
      \hline
      Алгоритм & 100 ч. & 1000 ч. & 5000 ч. & 10000 ч. & 100000 ч. \\
      \hline
      Использование сетки & 398 & 315 & 150 & 64 & 47 \\
      Полный перебор & 324 & 203 & 121 & 70 & 8 \\
      \hline
    \end{tabular}
  \end{center}
\end{table}

\begin{table}[H]
  \caption{\label{tab:snd:cpu}Комп. №2. Зависимость от типы процессора}
  \begin{center}
    \begin{tabular}{|c|c|c|c|c|c|}
      \hline
      Тип процессора & 100 ч. & 1000 ч. & 5000 ч. & 10000 ч. & 100000 ч. \\
      \hline
      Графический & 398 & 315 & 150 & 64 & 47 \\
      Центральный & 153 & 102 & 31 & 7 & $\approx{}0.4$ \\
      \hline
    \end{tabular}
  \end{center}
\end{table}


% написать как проводились вычисления fps
% различная визуализация
% n^2 и поиск ближайших
% сравнение по количеству частиц
% сравнение на разных gpu
% сравнение cpu и gpu

  \newpage

\section{Внедрение} % 3

Физика большого количества частиц отлично подходит для симуляции множества
природных явлений. Поведение воды, горение огня, физика ткани и т.д.
Находится применение в астрофизике, химии и биологии. Кроме того,
данный подход к вычислениям подходит не только для визуализации, но 
и для вычислений введенных пользователем данных. Например, для кластеризации 
поисковой выдачи в географических информационных системах.

Благодаря выбранной платформе, задача донести информацию до конечного пользователя
значительно упрощается. Нет необходимости устанавливать дополнительные плагины
и расширения для веб-браузера. Упрощается внедрение в уже существующие системы, 
так как большинство обучающих и экспертных систем используют веб-интерфейс.

\subsection{Учебные курсы}

Все большую популярность набирают системы онлайн-образования. К ним можно
отнести такие проекты, как Coursera \footnote{http://coursera.org}, mooc.org \footnote{http://mooc.org}
разрабатываемый компанией Google. Данные системы образования распространяются бесплатно 
и работают как веб-приложения.

Так же подобные визуальные представления могут использоваться при публикации работ студентов 
и ученых на сайтах высших учебных заведений и исследовательских центров.

\subsection{Визуальные представления в экспертных системах}

Разрабатываемое решение может использоваться для создания визуальных представлений
в экспертных системах, предназначенных для диагностики, мониторинга и проектирования
в сложных процессах и механизмах. Полученные визуализации помогут максимально наглядно
донести информацию до пользователя.

\subsection{Игровая индустрия}

Существует несколько примеров игр, которые используют вычислительный метод, описанный 
в документе для создания игрового процесса. Это значит что разработанный инструмент
подходит для создания браузерных HTML5 игр.

Последние версии мобильных браузеров уже добавили поддержку включения WebGL в
экспериментальном режиме. Это значит что на мобильных устройствах так же можно будет
использовать подобный подход для создания визуальных эффектов.

  \newpage
\section*{ЗАКЛЮЧЕНИЕ}

В рамках данной работы изучены возможные способы визуализации в веб-браузерах,
найден механизм просчета физики большого количества частиц в реальном времени.
Разработан инструмент для создания интерактивных визуальных представлений.
Разработан пример визуализации вычислительного метода гидродинамики сглаженных 
частиц для демонстрации производительности и работы инструмента.

Проведены тесты производительности. Полученные данные доказывают что визуализация
выполняется достаточно быстро, чтобы сохранялась интерактивность.

Реализация отвечает требованиям, поставленным в техническом задании:

\begin{itemize}
  \item скорость визуализации $>25$ кадров в секунду при $10000$ частиц;
  \item разработан механизм взаимодействия с пользователем;
  \item программа исполняется в веб-браузере без использования дополнительных 
    плагинов и расширений;
\end{itemize}

Для достижения цели были поставлены и решены следующие задачи:

\begin{itemize}
  \item изучение возможностей графических процессоров для визуализации
    и вычисления физических данных;
  \item описан программный интерфейс работы с WebGL для создания
    визуальных представлений;
  \item разработка способов визуализации частиц;
  \item разработка методов взаимодействия пользователя с симуляцией;
  \item реализация показательного демонстрационного примера;
\end{itemize}

Пояснительная записка отражает описание всех этапов создания инструмента,
от постановки задачи и описания структуры до ее реализации.
 
  \newpage

  \begin{thebibliography}{widest entry}
    \bibitem{webcl10} Khronos Releases WebCL 1.0 Specification -- \url{https://www.khronos.org/news/press/khronos-releases-webcl-1.0-specification}.
    \bibitem{raytracing02} Timothy, J.P. Ray tracing on programmable graphics hardware / T.J. Purcell, I. Buck, W.R. Mark, P. Hanrahan // ACM Transactions on Graphics (TOG). -- 2002. -- V. 21, №3. -- P. 703-712.
    \bibitem{gpu} Fatahalian, K. A closer look at GPUs / K. Fatahalian, M. Houston // Communications of the ACM. -- 2008. -- V. 51, №10. -- P. 50-57.
    \bibitem{hgpuw} Luebke, D. How GPUs Work / D. Luebke, G. Humphreys // IEEE Computer. -- 2007. -- V. 40, №2. -- P. 96-100.
    \bibitem{khr11} WebGL Specification -- \url{http://www.khronos.org/registry/webgl/specs/1.0/}.
    \bibitem{khr09} The OpenGL ES Shading Language \url{http://www.khronos.org/registry/gles/specs/2.0/GLSL\textunderscore ES\textunderscore Specification\textunderscore 1.0.17.pdf}.
    \bibitem{glsl} Rost, R.J. OpenGL Shading Language / R.J. Rost. -- Addison-Wesley Professional, 2006. -- 800 p.
    \bibitem{rtrender} Akenine-Möller, T. Real Time Rendering / T. Akenine-Möller, E. Haines, N. Hoffman. -- A K Peters/CRC Press, 2008. -- 1045 p.
    \bibitem{sphHarada} Harada, T. Smoothed Particle Hydrodynamics on GPUs / T. Harada, S. Koshizuka, and Y. Kawaguchi. -- 2007. -- 8 p.
      In Proc. of Computer Graphics International, pp. 63–70. Geneva: Computer Graphics Society, 2007.
    \bibitem{sphMon92} Monaghan, J.J. Smoothed Particle Hydrodynamics / J.J. Monaghan // Annual Review of Astronomy and Astrophysics. -- 1992. -- V. 30. -- P. 543–574.
    \bibitem{neisearch} Bayraktar, S. GPU-Based Neighbor-Search Algorithm for Particle Simulations / S. Bayraktar, U. Gudukbay, B. Ozguç // J. Graphics Tools. -- 2009. -- V. 14. -- P. 31-42.
  \end{thebibliography}
\end{document}
