\newpage
\section*{Введение}

%Проблема: есть некоторый описанный процесс . Задача сделать интерактивную визуализацию большого количества взаимодействующих объектов.
%Критерии: интерактивность -> реал-тайм, должно выполняться в браузере.
%Вопрос как выполнить это максимально рационально.
%Решение: Реализация основных алгоритмов, которые могут понадобиться при создании визуализации.
%Пример использования: визуализация пользовательских данных различных для каждого пользователя. При создании учебных курсов.  Создание визуализаций в экспертных системах. Кластеризация поисковой выдачи в ГИС.

%Графическое представление заметно способствует усваиванию информации.
%Современные компьютерные технологии помогают разрабатывать интерактивные и 
%наглядные изображения информации. Интерактивная визуализация информация -- подход
%к обработке информации в информационных системах, которая превращается в непрерывный 
%процесс взаимодействия с информацией через визуальное отображение.
%Приложения визуализации информации возникают в таких областях, как информационные
%системы и программное обеспечение, биологические науки, искусственный интеллект и 
%анализ пользовательских данных.

Визуальное представление информации -- это интерпретация числовой и текстовой
информации в виде графиков, структурных схем, таблиц, карт и т.д.
.... еще какая-нибудь хуита про интерактивную визуализацию
\linebreak \\
Несмотря на развитие современных подходов к интерактивному отображению, задача
визуализации систем, требующих оперативного взаимодействие большого количества
объектов, все еще остается не тривиальной. Критериями эффективности решения
являются качество и доступность итоговой визуализации, а также время, необходимое
для ее создание.
\linebreak \\
Развитие параллельных вычислений с использованием графических процессоров сделало 
возможным визуализацию сложных интерактивных представлений информации. 
\linebreak \\
Веб приложения являются удачным решением быстрого донесения информации до большого
количества пользователей. Поддержка браузерами Новых стандарты такие как HTML5 и WebGL 
позволила использовать преимущества вычислительных мощностей GPU в веб-приложениях.
\linebreak \\
В данной работе предлагается решение для упрощения создания визуальных представлений
в веб-приложениях. Идея состоит в реализации алгоритмов и написании упрощенного 
API для использования WebGL в качестве инструмента параллельных вычислений.
В качестве демонстрации использования данного решения реализована визуализация
вычислительного метода для симуляции жидкостей и газов -- гидродинамика сглаженных частиц.

%Легче воспринимать информацию с интерактивными примерами.
%Стоит задача как визуализировать процессы где требуется много вычислений.
%Современные компьютерная графика и постоянное развитие.
%Современные веб-технологии, кросс-платформенность и прочие плюхи веба.
%Использование webgl как инструмента для вычисления и визуализации.

