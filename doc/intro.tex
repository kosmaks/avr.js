\newpage
\numless{\section{Введение}}

Визуальное представление информации -- это интерпретация числовой и текстовой
информации в виде графиков, структурных схем, таблиц, карт и динамических изображений.
В данной работе рассмотрены способы интерактивной визуализации информации.
Интерактивная визуализация -- подход к обработке информации в информационных 
системах, которая превращается в непрерывный процесс взаимодействия с 
визуальным отображением. Приложения визуализации информации востребованы в таких 
областях, как информационные системы и программное обеспечение, биологические 
науки, искусственный интеллект и анализ пользовательских данных.

Несмотря на развитие современных подходов к интерактивному отображению, задача
визуализации систем, требующих оперативного взаимодействия большого количества
частиц (десятки и сотни тысяч), все еще остается нетривиальной. Критериями 
эффективности решения являются качество и доступность итоговой визуализации, 
а также время, необходимое для ее создания.

Развитие параллельных вычислений с использованием графических процессоров сделало 
возможным визуализацию сложных интерактивных представлений информации. 

Веб-приложения являются удачным решением быстрого донесения информации до большого
количества пользователей. Поддержка браузерами новых стандартов, таких как HTML5 и WebGL,  
позволила использовать преимущества вычислительных мощностей GPU в веб-приложениях.

Цель данной работы: поиск метода для эффективного создания визуальных представлений,
требующих большого количество взаимодействующих частиц,
в веб-приложениях. Создания инструмента для интерактивного отображения и разработки
демонстрационного примера. В качестве демонстрации использования данного решения реализована 
визуализация вычислительного метода для симуляции жидкостей и газов -- гидродинамика 
сглаженных частиц.

%Легче воспринимать информацию с интерактивными примерами.
%Стоит задача как визуализировать процессы где требуется много вычислений.
%Современные компьютерная графика и постоянное развитие.
%Современные веб-технологии, кросс-платформенность и прочие плюхи веба.
%Использование webgl как инструмента для вычисления и визуализации.

