\newpage
\section{Техническое задание}

Необходимо разработать инструмент для вычисления и создания визуальных 
представлений большого количества оперативно взаимодействующих частиц.

Разработать визуализацию симуляции жидкости методом гидродинамики сглаженных
частиц в целях демонстрации работы инструмента.

Программное обеспечение должно включать в себя следующее:

\begin{itemize}
  \item API для работы с массивами частиц и их характеристиками;
  \item Типы визуальных представлений:
    \begin{itemize}
      \item Представление точек с заданными характеристиками в трехмерном пространстве;
      \item Представление векторов в трехмерном пространстве;
      %\item Реконструкция поверхности методом шагающих кубиков;
    \end{itemize}
  \item Способы взаимодействия пользователя с симуляцией:
    \begin{itemize}
      \item Воздействие на частицы методом отбрасывания лучей;
      \item Смена угла обзора;
    \end{itemize}
\end{itemize}

Требования к ПО и демонстрационному примеру:

\begin{itemize}
  \item Должно выполняться в веб-браузере;
  \item Полученная визуализация должна быть интерактивной;
  \item Симуляция должна генерировать 25-60 кадров в секунд 
    при количестве частиц свыше 10000;
\end{itemize}

\subsection{Этап первый}

На данном этапе необходимо провести сравнительный анализ возможных способов визуализации
трехмерных изображений в веб-браузере. Посмотреть ситуации, когда необходим данный тип 
визуальных представлений, чтобы сделать максимально универсальный программный интерфейс.

Следующий щаг -- обзор способов вычисления таких объемов данных. Поиск оптимальных
алгоритмов для реализации демонстрационного примера.

Затем необходимо составить программный интерфейс исходя из собранных сведений.

Когда вся информация собрана, следует переходить к следующему этапу.

\subsection{Этап второй}

На этом этапе выполняется реализация технической части работы. Написание программного
интерфейса и его реализация. 

Далее разрабатываются перечисленные выше типы визуальных представлений. Реализуются
способы взаимодействия пользователя с симуляцией.

Третий шаг -- реализация алгоритмов демонстрационного примера.

После реализации и тестирования необходимо разработать инструкцию по использованию, а
также задокументировать исходный код.

